\documentclass[10pt,letterpaper]{article}
\include{settings}

% Define per-paper macros.
\newcommand{\withurl}[2]{{#1}\footnote{\texttt{#2}}}
\newcommand{\rulemajor}[1]{\section{#1}}
\begin{document}
\vspace*{0.2in}

\begin{flushleft}
{\Large
\textbf\newline{Ten Quick Tips for Running a Less Painful Meeting}
}
\newline
\\
{Greg~Wilson}\textsuperscript{1,*}
\\
\textbf{1} RStudio, Inc., Toronto, Ontario M4L 2T9
\\
\bigskip
* greg.wilson@rstudio.com
\end{flushleft}

\section*{Abstract}

FIXME

\section*{Author Summary}

FIXME

\section*{Introduction}

Most people are really bad at meetings: they don't have an agenda going in, they
don't take minutes, they waffle on or wander off into irrelevancies, they repeat
what others have said or recite banalities simply so that they'll have said
something, and they hold side conversations (which pretty much guarantees that
the meeting will be a waste of time).  Knowing how to run a meeting efficiently
is a core skill for anyone who wants to get things done. (Knowing how to take
part in someone else's meeting is just as important, but gets far less
attention—as a colleague once said, everyone offers leadership training, nobody
offers followership training.) The most important rules for making meetings
efficient are not secret, but are rarely followed:

\rulemajor{Decide if there actually needs to be a meeting.}

If the only purpose is to share information, have everyone send a brief email
instead, or better yet, add notes to a shared document so that people can find
the information later. Remember, you can read faster than anyone can speak: if
someone has facts for the rest of the team to absorb, the most polite way to
communicate them is to type them in.

\rulemajor{Create an agenda.}

If nobody cares enough about the meeting to write a point-form list of what's
supposed to be discussed, or if that list isn't circulated at least a day in
advance, the meeting probably doesn't need to happen.  (Note: ``the agenda is
all the open issues in our GitHub repo'' doesn't count.)

\textbf{Include timings in the agenda.}

Agendas also help you keep the meeting moving (as in, ``That's very interesting,
but we have five other topics to get through in the next fifteen minutes, so
could you please make your point?''), so include the time to be spent on each
item in the agenda.  Your first estimates with any new group will be wildly
optimistic, so revise them upward for subsequent meetings.

\textbf{Prioritize.}

Every meeting is a micro-project, so work should be prioritized in the same way
that it is for other projects: things that will have high impact but take little
time should be done first, and things that will take lots of time but have
little impact should be skipped.

\textbf{End early.}

If your meeting is scheduled for 10:00-11:00, you should aim to end at 10:55 to
give people time to get where they need to go next.

\rulemajor{Have clear rules for making decisions.}

If more than half a dozen people are involved in decision-making, consider
adopting Martha's Rules~\cite{Minahan1986}.  (If you think you need Robert's
Rules of Order, you should probably get some training in how to run and
participate in meetings.)

\begin{enumerate}

\item
  Before each meeting, anyone who wishes may sponsor a proposal.
  Proposals must be circulated at least 24 hours before a meeting
  in order to be considered at that meeting, and must include:

  \begin{itemize}
  \item a one-line summary (the subject line of the issue);
  \item the full text of the proposal;
  \item any required background information;
  \item pros and cons; and
  \item possible alternatives.
  \end{itemize}

\item
  A quorum is established in a meeting if half or more of voting members are
  present.

\item
  Once a person has sponsored a proposal, they are responsible for it.  The
  group may not discuss or vote on the issue unless the sponsor or their
  delegate is present.  The sponsor is also responsible for presenting the item
  to the group.

\item
  After the sponsor presents the proposal, a \emph{sense vote} is cast for the
  proposal prior to any discussion:

  \begin{itemize}
  \item Who likes the proposal?
  \item Who can live with the proposal?
  \item Who is uncomfortable with the proposal?
  \end{itemize}

\item
  If all or most of the group likes or can live with the proposal, it is
  immediately moved to a formal vote with no further discussion.

\item
  If most of the group is uncomfortable with the proposal, it is postponed for
  further rework by the sponsor.

\item
  If some members are uncomfortable they can briefly state their objections.  A
  timer is then set for a brief discussion moderated by the facilitator.  After
  10 minutes or when no one has anything further to add (whichever comes first),
  the facilitator calls for a yes-or-no vote on the question: ``Should we
  implement this decision over the stated objections?''  If a majority votes
  "yes" the proposal is implemented.  Otherwise, the proposal is returned to the
  sponsor for further work.

\end{enumerate}

\rulemajor{Put someone in charge.}

``In charge'' means keeping the meeting moving, glaring at people who are
muttering to one another or checking email, and telling people who are talking
too much to get to the point.  It does *not* mean doing all the talking; in
fact, whoever is in charge will usually talk less than anyone else, just as a
referee usually kicks the ball less often than the players.  One way to ensure
this is to give the moderator a stuffed animal or something else to hold up when
they're speaking on their own behalf rather than moderating.

\textbf{Keep a backlog.}

If anything comes up that looks like it will take the conversation into the
weeds, write it down on a sticky note and put it on the table (or the wall
behind the moderator, or wherever else is convenient), and then deal with it at
the end of the meeting.  This works better than having the moderator take a
note, since everyone can see the backlog piling up and compare it against the
time remaining.

\rulemajor{Require politeness.}

No one gets to be rude, no one gets to ramble, and if someone goes off topic,
it's the chair's job to say, ``Let's discuss that elsewhere.''  If your project
has a code of conduct (which it should), the moderator should remind people of
it at the start of each meeting.

\textbf{No technology.}

Insist that everyone put their phones, tablets, and laptops into politeness mode
(i.e., closes them).  If this is too stressful, let participants hang on to
their electronic pacifiers, but turn off the network so that they really *are*
using them just to take notes or check the agenda.  Before insisting on this,
though, please check if people need their devices for accessibility reasons.

\textbf{No interruptions.}

Participants should raise a finger, put up a sticky note, or make one of the
other gestures people make at high-priced auctions instead if they want to speak
next.  If the speaker doesn't notice you, the person in charge ought to.

\rulemajor{Record minutes.}

Someone other than the chair should take point-form notes about the most
important pieces of information that were shared, and about every decision that
was made or every task that was assigned to someone.

\textbf{Take notes.}

While other people are talking, participants should take notes of questions they
want to ask or points they want to make.  (You'll be surprised how smart it
makes you look when it's your turn to speak.)

\rulemajor{Post the minutes.}

As soon as the meeting is over, the minutes should be circulated (e.g., emailed
to everyone or posted to a wiki):

\begin{itemize}

\item
  \textbf{People who weren't at the meeting can keep track of what's going on.}
  You and your fellow students all have to juggle assignments from several other
  courses while doing this project, which means that sometimes you won't be able
  to make it to team meetings.  A wiki page, email message, or blog entry is a
  much more efficient way to catch up after a missed meeting or two than asking
  a team mate, ``Hey, what did I miss?''

\item
  \textbf{Everyone can check what was actually said or promised.} More than
  once, I've looked over the minutes of a meeting I was in and thought, ``Did I
  say that?'' or, ``Wait a minute, I didn't promise to have it ready then!''
  Accidentally or not, people will often remember things differently; writing it
  down gives team members a chance to correct mistaken or malicious
  interpretations, which can save a lot of anguish later on.

\item
  \textbf{People can be held accountable at subsequent meetings.} There's no
  point making lists of questions and action items if you don't follow up on
  them later.  If you're using a ticketing system, the best thing to do is to
  create a ticket for each new question or task right after the meeting, and
  update those that are being carried forward.  That way, your agenda for the
  next meeting can start by rattling through a list of tickets.

\end{itemize}
  
\rulemajor{Manage ``That Guy''}

Some people are so used to the sound of their own voice that they will insist on
talking half the time no matter how many other people are in the room. One way
to combat this is to give everyone three sticky notes at the start of the
meeting. Every time they speak, they have to take down one sticky note. When
they're out of notes, they aren't allowed to speak until everyone has used at
least one, at which point everyone gets all of their sticky notes back. This
ensures that nobody talks more than three times as often as the quietest person
in the meeting, and completely changes the dynamics of most groups: people who
have given up trying to be heard because they always get trampled suddenly have
space to contribute, and the overly-frequent speakers quickly realize just how
unfair they have been.

Another useful technique is \emph{interruption bingo}. Draw a grid and label
the rows and columns with the participants' names. Each time someone interrupts
someone else, add a tally mark to the appropriate cell. Halfway through the
meeting, take a moment to look at the results.  In most cases, you will see that
one or two people are doing all of the interrupting, often without being aware
of it. After that, saying, ``All right, I'm adding another tally to the bingo
card,'' is often enough to get them to throttle back.

\section*{Conclusion}

This isn't the only way to run a meeting: different cultures have different
conventions, and every group will evolve its own practices.  FIXME

\bibliography{10-meeting}

\end{document}
